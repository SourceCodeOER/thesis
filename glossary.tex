% J'ai choisi la "long3col-booktabs" mais il y en a d'autres :
% https://www.dickimaw-books.com/gallery/glossaries-styles/
\setglossarystyle{long3col-booktabs}

% Les entrées du glossaire
% https://www.overleaf.com/learn/latex/glossaries

\newglossaryentry{api}
{
    name={API},
    description={
        Acronyme anglais pour "Application Programming Interface".
        Solution informatique permettant à des applications de communiquer et d'échanger des données ou des services entre elles.
    }
}

\newglossaryentry{cli}
{
    name={CLI},
    description={
        Acronyme anglais pour "Command Line Interface".
        Solution informatique permettant à un utilisateur de communiquer avec sa machine de manière textuelle.
    }
}

\newglossaryentry{rest}
{
    name={REST},
    description={
        Acronyme anglais pour "REpresentational State Transfer".
        Style d'architecture définissant des contraintes pour concevoir des services communiquant par le web.
    }
}

\newglossaryentry{frontend}
{
    name={Frontend},
    description={
        Il s'agit de la partie "émergée" de l'application ( qui peut être comparé à un iceberg ). Les utilisateurs finaux interagissent uniquement avec celle-ci.  
    }
}

\newglossaryentry{backend}
{
    name={Backend},
    description={
        Il s'agit de la partie "immergée" de l'application ( qui peut être comparé à un iceberg ). Le \Gls{frontend} et le \Gls{cli} interagissent avec celle-ci.
    }
}

\newglossaryentry{oas}
{
    name={OAS},
    description={
        Acronyme anglais pour "OpenAPI Specification".
        Il s'agit d'une manière de décrire des \Gls{api} de type \Gls{rest} qui soit à la fois compréhensible pour les ordinateurs et les humains, sans que ces derniers ne doivent inspecter le code source ou jouer aux devinettes.
    }
}

% TODO ajouter la source ( faut que je choisis le style de bibliographie )
% https://fr.unesco.org/themes/tic-education/rel
\newglossaryentry{oer}
{
    name={OER},
    description={
        Acronyme anglais pour "Open Educational Resources". ( dans sa version française, "REL" pour "Ressources éducatives libres" ).
        La définition officielle de UNESCO\cite{UNESCO} les désigne comme étant "des matériaux d’enseignement, d'apprentissage ou de recherche appartenant au domaine public ou publiés avec une licence de propriété intellectuelle permettant leur utilisation, adaptation et distribution à titre gratuit".
    }
}

\newglossaryentry{crud}
{
    name={CRUD},
    description={
        Acronyme anglais pour "Créer"(\textbf{C}reate), "Lire" (\textbf{R}ead), "Mettre à jour"(\textbf{U}pdate), "Supprimer"(\textbf{D}elete). Il décrit les opérations de base pour la persistance des données (comme les bases de données). Un lien avec \Gls{rest} peut être fait par les méthodes HTTP GET, POST, PUT, DELETE respectivement.
    }
}
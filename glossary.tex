% les styles possibles :
% https://www.dickimaw-books.com/gallery/glossaries-styles/
\setglossarystyle{long-booktabs}
% Avant c'était
% \setglossarystyle{long3col-booktabs}

% Les entrées du glossaire
% https://www.overleaf.com/learn/latex/glossaries

% Pour le singulier : \gls
% Pour le pluriel : \glspl
% (https://tex.stackexchange.com/questions/128396/defining-plural-form-for-glossaries-entries)

% Pour les majuscules, c'est un G majuscule
% Pour le singulier : \Gls
% Pour le pluriel : \Glspl
% (http://ctan.cs.uu.nl/macros/latex/contrib/glossaries/glossariesbegin.html)

\newglossaryentry{api}
{
    name={API},
    description={
        Acronyme anglais pour "Application Programming Interface".
        Solution informatique permettant à des applications de communiquer et d'échanger des données ou des services entre elles.
    }
}

\newglossaryentry{cli}
{
    name={CLI},
    description={
        Acronyme anglais pour "Command Line Interface".
        Solution informatique permettant à un utilisateur de communiquer avec sa machine de manière textuelle.
    }
}

\newglossaryentry{rest}
{
    name={REST},
    description={
        Acronyme anglais pour "\textbf{RE}presentational \textbf{S}tate \textbf{T}ransfer".
        Style d'architecture logicielle avec des contraintes/conventions techniques pour concevoir des services communiquant par le web.
    }
}

\newglossaryentry{frontend}
{
    name={Frontend},
    description={
        Il s'agit de la partie "émergée" de l'application (comparable à un iceberg). C'est l'interface graphique avec laquelle les utilisateurs finaux interagissent.  
    }
}

\newglossaryentry{backend}
{
    name={Backend},
    description={
        Il s'agit de la partie "immergée" de l'application (comparable à un iceberg) : notre \Gls{api}.
    }
}

\newglossaryentry{oas}
{
    name={OAS},
    description={
        Acronyme anglais pour "OpenAPI Specification".
        Il s'agit d'une manière de décrire des \Gls{api} de type \Gls{rest} qui soit à la fois compréhensible pour les ordinateurs et les humains, sans que ces derniers ne doivent inspecter le code source ou jouer aux devinettes.
    }
}

\newglossaryentry{oer}
{
    name={OER},
    description={
        Acronyme anglais pour "Open Educational Resources" (dans sa version française, "REL" pour "Ressources éducatives libres").
        La définition officielle de l'UNESCO\cite{UNESCO} les désigne comme étant "des matériaux d’enseignement, d'apprentissage ou de recherche appartenant au domaine public ou publiés avec une licence de propriété intellectuelle permettant leur utilisation, adaptation et distribution à titre gratuit".
    }
}

\newglossaryentry{crud}
{
    name={CRUD},
    description={
        Acronyme anglais pour "Créer"(\textbf{C}reate), "Lire" (\textbf{R}ead), "Mettre à jour"(\textbf{U}pdate), "Supprimer"(\textbf{D}elete). Il décrit les opérations de base pour la persistance des données (comme les bases de données). Un lien avec \Gls{rest} peut être fait par les méthodes HTTP POST, GET, PUT, DELETE respectivement.
    }
}
\newglossaryentry{QQOQCCP}
{
    name={QQOQCCP},
    description={
        Acronyme francophone pour "\textbf{Q}ui ? \textbf{Q}uoi ? \textbf{O}ù ? \textbf{Q}uand ? \textbf{C}omment ? \textbf{C}ombien ? \textbf{P}ourquoi ?". Celui-ci donne un cadre de questions à poser, notamment pour l'analyse.
    }
}

\newglossaryentry{tagCat}
{
    name={Catégorie de mots-clés},
    text={catégorie de mots-clés},
    plural={catégories de mots-clés},
    description={
        Rubrique contenant des \glspl{tag} de même nature. 
    }
}

\newglossaryentry{tag}
{
    name={Mot-clé},
    text={mot-clé},
    plural={mots-clés},
    description={
        Mot (ou groupe de mots) permettant de caractériser une ou plusieurs \gls{fiche}(s).
        Un mot clé est toujours lié à une \gls{tagCat}.
    }
}

\newglossaryentry{fiche}
{
    name={Fiche},
    text={fiche},
    plural={fiches},
    description={
        Document décrivant une \gls{resinfo} ainsi que les \glspl{metadata} liées à elle. Parmi les informations renseignées, on trouve notamment des \glspl{tag}.
    }
}

\newglossaryentry{metadata}
{
    name={Métadonnée},
    text={métadonnée},
    plural={métadonnées},
    description={
        Type de données permettant de caractériser et structurer des \glspl{resinfo}.
    }
}

\newglossaryentry{resinfo}
{
    name={Ressource informatique},
    text={ressource informatique},
    plural={ressources informatiques},
    description={
        Désigne tout matériel ayant pour attrait le domaine de l'informatique. Exemple : des exercices de programmation, des tutoriels en informatique, des algorithmes, des slides, ...
    }
}
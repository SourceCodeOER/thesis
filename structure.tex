% Fichier qui contiendra tout la structure du mémoire

% Use french everywhere
\selectlanguage{french}

% Pour include d'autres fichiers
% ./ si folder courant
%\import{folder}{fichier.tex}

% Numérotation en chiffres romains ( I , II )
\frontmatter

% Table des matières + autres tables
\import{sections/}{tables}

% Remerciements ( Un classique )
\import{sections/}{remerciements}

% Le Préambule
\import{sections/}{preambule}

% Numérotation en chiffres arabes ( 1, 2 , )
\mainmatter

% Pour l'instant, je garde les autres titres des chapitres ici car on va sûrement en débattre / changer des trucs

% WARNING : comme expliqué au dessus, les lecteurs sont économes de leur temps : ils ne lisent bien souvent que l'introduction, la conclusion et quelques passages par ci par là . Du coup, bien travailler l'intro et la ccl
\chapter{Introduction}

\chapter{Contexte}
\section{Situation actuelle}
\section{Analyse des plate-formes existantes}
\section{Défis à relever}

\chapter{Organisation du travail}
\section{Méthodologie}
\section{Planning}
% Une illustration résumée est appréciable à un millier de mots ^^
\section{Répartition des tâches}

\chapter{Analyse initiale}
% Partie dans laquelle on explique les features / requirements attendues
% On y trouve sans doute :

\section {Fonctionnalités}
% - diagrammes de cas d'utilisations ( + visuel qu'une liste de user stories )
% - explications des différents rôles dans l'application ( guest, user, admin )

\section {Contraintes}
% Expliquer les contraintes :
%    - Maintenabilité / Extensibilité  du code
%    - Gestion simple du système ( quand Mens disait que les admins ne devaient pas avoir à toucher à la DB pour faire x ou y truc )
%    - Interface orienté vers l'UX ( une jolie instruction UI VS UX
%    - etc...

\section{Langue de programmation}
% Explication du choix de Node.Js : 
% - 
% - pourquoi est ce intéressant à choisir 
%    - expliquer les concepts de taches async avec une illustration

\chapter{Application}
\section{Partie API}
\subsection{Choix technologiques}
\subsection{Architecture du projet}
\subsection{Points clés de l'implémention}
% Pas besoin d'expliquer tout , seulement l'essentiel : 
% - OpenAPI (introduit dans les choix) mais montrer la validation/doc/etc
% - Les middelwares ( securité, validation, etc... )
% - ORM 


\section{Partie Client}
\subsection{Choix technologiques}
\subsection{Architecture du projet}
% TODO : Je me doute que tu as ta propre manière d'aborder la chose ici, à toi de jouer ;)
% Pour moi, on doit y retrouver : 
% - des captures d'écrans montrant les principales fonctionnalités :
%    - recherche ( un exemple avec multi critères )
%    - création d'exercice
%    - la modération d'exercices par un admin ( la main page avec plusieurs)
% - avec les captures, des explications sur la conception des pages
% ( Par exemple, "Afin de résoudre les problèmes UX que nous avons constatés avec les autres plateformes, nous avons ...")
% - ETC 


% Truc qu'Olivier a tellement insisté
\chapter{Analyse critique}
\section{État du projet}
\section{Métriques} 
% Petits rappels de SQA ^^ 
% LOC ( lines of code ) / Code coverage 
\section{Idées de fonctionnalités futures}

\chapter{Conclusion}

% juste avant la bibliographie et les annexes
\backmatter

% Bibliographie
\import{sections/}{bibliographie}

% Annexes
% \appendix
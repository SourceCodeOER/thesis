% Fichier qui contiendra tout la structure du mémoire

% Use french everywhere
\selectlanguage{french}

% Pour include d'autres fichiers
% ./ si folder courant
%\import{folder}{fichier.tex}

% Numérotation en chiffres romains ( I , II )
\frontmatter

% Table des matières + autres tables
\import{sections/}{tables}

% Remerciements ( Un classique )
\import{sections/}{remerciements}

% Le Préambule
\import{sections/}{preambule}

% Numérotation en chiffres arabes ( 1, 2 , )
\mainmatter

% Les différents chapitres ( tous dans un dossier) 
% avec comme entrypoint un fichier index.tex
% ( à part cela, on peut structurer comme on veut le tout, comme je l'ai fait avec le chapitre "solution" )
\import{sections/chapters/introduction/}{index}
\import{sections/chapters/problematique/}{index}
\import{sections/chapters/approche/}{index}
\import{sections/chapters/cahierDesCharges/}{index}
\import{sections/chapters/solution/}{index}

% Truc qu'Olivier a tellement insisté
\chapter{Analyse critique}
\section{Validation}
\section{Métriques}
% Petits rappels de SQA ^^ 
% LOC ( lines of code ) / Code coverage
\section{État du projet}
% Fonctionnalité
% Etat du prototype

 
\chapter{Pour aller plus loin}

\chapter{Conclusion}

% juste avant la bibliographie et les annexes
\backmatter

% Bibliographie
\import{sections/}{bibliographie}

% Annexes
% \appendix
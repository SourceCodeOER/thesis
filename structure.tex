% Fichier qui contiendra tout la structure du mémoire

% Use french everywhere
\selectlanguage{french}

% Pour include d'autres fichiers
% ./ si folder courant
%\import{folder}{fichier.tex}

% Numérotation en chiffres romains ( I , II )
\frontmatter

% Table des matières + autres tables
\import{sections/}{tables}

% Le glossaire
% Pour forcer l'affichage de tout ( mais pas idéal car cela n'affiche pas les bonnes pages pour les termes ): https://tex.stackexchange.com/a/492011/129702
%\glsaddall
\printglossary

% Remerciements ( Un classique )
\import{sections/}{remerciements}

% Le Préambule
\import{sections/}{preambule}

% Numérotation en chiffres arabes ( 1, 2 , )
\mainmatter

% Petite précision : si \import est hs dans ta structure nested, essaye un \include ( mieux qu'un \input qui peut casser les pieds dans certains cas )
% https://tex.stackexchange.com/a/32058/129702
\import{sections/chapters/introduction/}{index}
\import{sections/chapters/problematique/}{index}
\import{sections/chapters/approche/}{index}
\import{sections/chapters/cahierDesCharges/}{index}


% Truc qu'Olivier a tellement insisté
\chapter{Analyse critique}
\section{Validation}
\label{section:validation}
\section{Métriques}
% Petits rappels de SQA ^^ 
% LOC ( lines of code ) / Code coverage
\section{État du projet}
% Fonctionnalité
% Etat du prototype

 
\chapter{Pour aller plus loin}

\chapter{Conclusion}

% juste avant la bibliographie et les annexes
\backmatter

% Bibliographie
\import{sections/}{bibliographie}

% Annexes
% Explications : https://texfaq.org/FAQ-appendix
\begin{appendices}
    % Ici on includera que des pdf et/ou images 
% ( car trop compliqué de gérer plusieurs bibliographies dans le même document en latex )

% il devrait pas y avoir de soucis pour conserver les titres avec les pdfs
% https://tex.stackexchange.com/questions/440046/title-of-subsection-on-included-multi-page-pdf

\chapter{Analyse Bibliographique}
\label{annexe:AnalyseBiblio}

TODO includegraphics pdf
\end{appendices}
\includepdf[
    pages={1},
    offset=0cm -1cm,
    width=\textwidth,
    pagecommand={
        \fboxrule0.4pt
        \chapter{Analyse Bibliographique}
        \label{annexe:AnalyseBiblio}
    },
]{sections/annexes/analyseBibliographique/Analyse_Bibliographique.pdf}

\includepdf[
    pages=2-,
    % Add page number below 
    pagecommand={\thispagestyle{plain}},
    width=1.1\textwidth,
    % To have items into list of figures / tables / listings
    % WARNING : As explained by the doc, this option is experimental (it MAY change but I don't think so : many people used that)
    % SECOND WARNING : the page number argument is the real position from starting point !!
    addtolist={
        5, figure, {Les 15 éléments du Dublin Core}, fig:dcListElements,
        6, figure, {Les 9 catégories d'éléments du LOM}, fig:LOMTable,
        7, figure, {Représentation hiérarchique du LOM}, fig:LOMChart,
        7, figure, {Correspondance des éléments entre le Dublin Core et le LOM}, fig:mappingDcLom,
        8, table, {Les 5 types de données du LOM}, tab:typeLOM,
        9, figure, {MLR - Spécification d’un élément de données}, fig:mlr,
        9, figure, {MLR - Exemple de spécification d’un élément de données}, fig:mlr2,
        10, figure, {Représentation simplifiée du MLR}, fig:mlr3,
        11, figure, {Exemple de taxonomie pour les exercices de programmation}, fig:taxonomieEx,
        12, figure, {Liste des sous-domaines de l'informatique selon le CC2005}, fig:AllDomainInInfo,
        13, figure, {Exemple de hiérarchie de compétences}, fig:comptenceTree,
        13, figure, {Catégories d'exercices informatiques}, fig:CatExoList,
        15, table, {Éléments de base de notre norme}, tab:baseNorm,
        16, table, {Éléments d'un mot-clé de notre norme}, tab:KeyElementsNorm,
        16, table, {Quelques éléments additionnels de notre norme}, tab:ExtendedElementsNorm,
        17, lstlisting, {Exemple d'une ressource information avec notre norme}, code:exampleJson
    }
]{sections/annexes/analyseBibliographique/Analyse_Bibliographique.pdf}
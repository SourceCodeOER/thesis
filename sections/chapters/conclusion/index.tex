\chapter{Conclusion}

Ce mémoire a pour but principal d’apporter une solution à la problématique des ressources informatiques par le biais d’une plateforme web devant réaliser des objectifs ambitieux, notamment en matière d’expérience utilisateur et de référencement, avec une durée de développement restreinte pour un projet de cette ampleur. \\

En dépit de sa jeunesse, notre solution Open Source semble avoir rencontré ces critères, car nous avons pu la faire valider autant par un panel diversifié d’utilisateurs qu’au niveau de sa réalisation technique à proprement parler (tests, métriques sur la qualité du code ...). En outre, des processus automatisés (cf. section 5.4.3) permettent notamment le déploiement aisé et rapide d’un logiciel fonctionnel : il n’est donc pas exclu d’espérer une reprise ultérieure de Source Code ou à défaut la création d’une nouvelle solution inspirée de la nôtre. \\

Sur le niveau personnel, ce projet a été une source intarissable d’expériences. Celui-ci a été le plus conséquent que nous ayons pu connaître lors de notre parcours universitaire et nous avons beaucoup appris sur l’organisation du travail et sur l’utilisation de nouvelles technologies et outils de développement. Nous restons cependant conscients que notre solution reste perfectible et critiquable dans son état actuel (cf. 6.1 / 7.1), ce pour quoi nous avons réalisé un roadmap de développement prenant en compte les diverses remarques et potentielles fonctionnalités. \\

Finalement, c'est avec un certain sentiment du devoir accompli, mais aussi d'un désir inassouvi d'améliorations possibles à tout niveau que nous concluons ce présent mémoire. 

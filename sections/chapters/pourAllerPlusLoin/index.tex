\chapter{Pour aller plus loin}
\label{chapter:pourAllerPlusLoin}

Ce chapitre tente de rassembler tous les futurs travaux qui pourront être apportés à \texttt{Source Code}.\\

Durant notre année académique, nos promoteurs ont gardé un œil sur le projet afin de nous proposer quelques pistes pour améliorer l'application.
Parallèlement, nous avons pu rencontrer \textit{Christine Jacqmot}, membre du Louvain Learning Lab, qui nous a partagé son expérience des \gls{oer} avec quelques conseils pour notre plateforme. 
Grâce à ces précieux avis, à notre prise de recul et aux remarques récoltées lors de la séance de validation (cf. \ref{section:validation}) avec des utilisateurs, nous avons dressé une liste des améliorations possibles pour \texttt{Source Code}.

\section{Liste des améliorations}

\begin{itemize}
    \item Améliorer le système de validation des \glspl{resinfo} en ajoutant un système de tickets (issues) comme GitHub. Cela permettra au créateur de la ressource de communiquer avec les administrateurs en sachant ce qui ne va pas avec sa \gls{fiche}.
    \item Faciliter la modération de \glspl{resinfo} en automatisant la recherche de ressources similaires à d'autres ou non pertinentes (contenu injurieux, mal écrit,...).
    \item Ajouter des métriques supplémentaires pour les \glspl{resinfo} :
    \begin{enumerate}
        \item Nombre de fois où la \gls{resinfo} a été consultée.
        \item Nombre de fois où l'archive zip a été téléchargée.
    \end{enumerate}
    % \item Améliorer le référencement d'une \gls{resinfo} avec les URI et DOI.
    \item Prévoir l'intégration avec de multiples plateformes dont Inginious, Google Scholar pour améliorer le référencement des \glspl{resinfo}.
    \item Ajouter le multi-upload de fichier dans le formulaire de création/modification d'une \gls{resinfo}.
    \item Ajouter le preview de fichier quand c'est possible (image, pdf, word,...) dans le formulaire de création/modification de \glspl{resinfo}.
    \item Avoir la possibilité de rendre certains favoris publics afin de les partager avec les autres et les sauvegarder dans ses propres favoris.
    \item Ajouter la compatibilité avec les normes Dublin Core et LOM (cf. annexe \ref{annexe:AnalyseBiblio}, pratique pour l'export sous un certain format et l'import capable de convertir les formats). Cela permettra à \texttt{Source Code} de s'ouvrir à d'autres plateformes qui utilisent d'autres normes.
    \item Gestion de la recherche approximative dans la bibliothèque avec propositions de recherche.
    \item Améliorer la lisibilité des \glspl{tag} dans le formulaire de création de \glspl{resinfo} ou de favoris. Il faut pouvoir distinguer à quelle \gls{tagCat} un \gls{tag} appartient, sans devoir rechercher dans le panneau de filtres.
    \item Améliorer l'éditeur de texte pour la création/modification d'une \gls{resinfo}.
    \begin{enumerate}
        \item Créer un bloc de code directement en sélectionnant plusieurs lignes au lieu de créer un bloc puis coller le code dedans.
        \item Agrandir la boîte d'édition de texte en fonction de l'écran.
    \end{enumerate}
    \item Ajouter un dark mode.
    \item Ajouter une fonction de mot de passe oublié.
    \item Ajouter d'autres possibilités d'authentification (par exemple réseaux sociaux ou autres).
    \item Pouvoir résoudre certains exercices directement depuis la plateforme avec l'automatisation de la correction du code.
    \item Créer un espace de discussion autour d'une \gls{resinfo}. Cela peut être utile pour poser des questions sur la résolution d'un exercice.
    \item Ajouter un support complet du responsive sur la plateforme (cf. section \ref{sec:ContraintesCdc})
    \item Ajouter une recherche avancée dans la bibliothèque avec des conditions (NOT, AND, OR). Cela a déjà été intégré côté \gls{backend}, il reste donc à l'ajouter dans le \gls{frontend}.
    \item Pouvoir modifier son profil et consulter certaines statistiques comme :
    \begin{enumerate}
        \item La moyenne totale des \glspl{resinfo} créées
        \item Le nombre de \glspl{resinfo} créées
    \end{enumerate}
    \item Intégrer GraphQL à l'\gls{api} pour permettre de charger les données que l'on souhaite, sans forcément multiplier les endpoints en \Gls{rest}.
    \item Mettre en place des limites d'utilisation sur l'\gls{api} (ex. un utilisateur ne pourrait pas importer plus que n \glspl{fiche} par quart d'heure).
\end{itemize}

Nous avons pris plaisir à développer cette plateforme web pour une cause qui nous concerne dans le domaine informatique : le partage, comme pour les projets open source. Durant l'année académique, le temps fut notre ennemi, mais nous avons fait en sorte de fournir une plateforme fonctionnelle, à défaut qu'elle ne soit pas totalement complète.\\

La liste évoquée plus haut peut donc être considérée comme un prochain roadmap de développement. Il y a encore pas mal à réaliser, mais nous insistons sur le fait que ce projet est open source et utilise des technologies connues (cf. \ref{section:choixTechnologiques}) afin que ça ne devienne pas une barrière au développement.
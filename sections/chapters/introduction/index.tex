% WARNING : comme expliqué au dessus, les lecteurs sont économes de leur temps : ils ne lisent bien souvent que l'introduction, la conclusion et quelques passages par ci par là . Du coup, bien travailler l'intro et la ccl
\chapter{Introduction}

\paragraph{} Dans une époque dominée par la technologie, la programmation se révèle aujourd’hui être un atout irréfutable pour se bâtir une perspective professionnelle durable. Compte tenu de cela, de plus en plus d’institutions pédagogiques intègrent une section informatique afin d’enseigner cette discipline prometteuse (comme l'illustre cet article de la Fédération Wallonie-Bruxelles \cite{MAG_PROG}). Au-delà de l’aspect théorique, la programmation se découvre aussi par la pratique, ce pourquoi les chargés de cours constituent dans leur cursus un \textit{catalogue} d’exercices de programmation pour que les élèves puissent les résoudre au cours de l'année. 

\paragraph{} Créer des énoncés est une activité \textit{chronophage} et bien souvent, nous ne faisons que réinventer la roue car une autre personne a très certainement déjà envisagé ce même exercice par le passé.

\paragraph{} Dans une ère où l’information se trouve à quelques clics de souris, pourquoi ne pas proposer un service entièrement basé sur la \textit{contribution d’un catalogue « Open Source »} ? C’est donc ici que démarre notre projet : \texttt{Source Code}.

\paragraph{} A l’instar des \textit{Open Educational Resources (\Gls{oer})}, notre plateforme offre la possibilité à des équipes pédagogiques de collaborer sur la problématique de référecement et de partage des exercices. Cette dernière consiste à référencer des ressources informatiques, en permettant à un public varié de profiter de toutes contributions. 
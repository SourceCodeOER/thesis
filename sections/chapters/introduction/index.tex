% WARNING : comme expliqué au dessus, les lecteurs sont économes de leur temps : ils ne lisent bien souvent que l'introduction, la conclusion et quelques passages par ci par là . Du coup, bien travailler l'intro et la ccl
\chapter{Introduction}

\section*{Contexte}

Dans une époque dominée par la technologie, la programmation se révèle aujourd’hui être un atout irréfutable pour se bâtir une perspective professionnelle durable. Compte tenu de cela, de plus en plus d’institutions pédagogiques intègrent une section informatique afin d’enseigner cette discipline prometteuse (comme l'illustre cet article de la Fédération Wallonie-Bruxelles \cite{MAG_PROG}). Au-delà de l’aspect théorique, la programmation se découvre aussi par la pratique, ce pour quoi les chargés de cours constituent dans leur cursus un \textit{catalogue} d’exercices de programmation pour que les élèves puissent les résoudre au cours de l'année.\\

Créer des énoncés est une activité chronophage et bien souvent, nous ne faisons que réinventer la roue, car une autre personne a très certainement déjà envisagé ce même exercice par le passé.
Dans une ère où l’information se trouve à quelques clics de souris, pourquoi ne pas proposer un service entièrement basé sur la \textit{contribution d’un catalogue « Open Source »} ? C’est donc ici que démarre notre projet : \texttt{Source Code}.\\

\section*{Problématique}

Ce mémoire se concentre exclusivement sur la problématique de partage de \glspl{resinfo}.
Cette dernière implique un effort considérable dans la création d'une bibliothèque de \glspl{resinfo}, car celle-ci constitue les fondations de la plateforme. En effet, partant de la bibliothèque, d'autres éléments interdépendants seront nécessaires pour assurer sa qualité et sa praticité (critères de recherche, modération, référencement ...).

\section*{Motivation}

Ce mémoire, réalisé dans le cadre de notre formation, nous permet de mettre en oeuvre ce qui a été enseigné durant notre cursus académique. Partant de rien, si ce n'est qu'avec la vision et les objectifs de nos promoteurs, nous devons en toute autonomie sortir de notre zone de confort pour apporter une solution au problème des \glspl{resinfo} partagées.\\

Projet axé développement oblige, nous sommes donc confrontés à choisir et utiliser des technologies pour créer la plateforme. En ce sens, nous apprenons au travers de ce mémoire à parfaire nos compétences en développement et à nous familiariser avec de nouveaux outils de développement.
Enfin, nous espérons que notre travail puisse être profitable au domaine des ressources partagées. Afin de renforcer cette volonté, ce projet est totalement Open Source.

\section*{Objectif}

À l’instar des \textit{Open Educational Resources (\Gls{oer})}, notre plateforme souhaite offrir la possibilité à des équipes pédagogiques de collaborer sur la problématique de partage d'exercices. Cette dernière consiste à référencer des \glspl{resinfo}, en permettant à un public varié de profiter de toutes contributions.\\

\section*{Approche}

Avant de démarrer la réalisation de la plateforme web, nous devions impérativement définir les besoins d'une plateforme de partage de \glspl{resinfo}. Cette phase d'analyse s'est déroulée en deux temps.\\

Le premier temps consistait à analyser des plateformes similaires existantes afin d'en extraire des éléments visuels ou fonctionnels pertinents pour ce projet tandis que pour le second, il nous fallait effectuer d'une part, une analyse fonctionnelle regroupant l'organisation de la plateforme et les fonctionnalités à intégrer avec la méthode \textit{\textbf{MoSCoW}}, et d'autre part une analyse non fonctionnelle afin d'y relater les diverses contraintes du projet, les critères de sécurité et de maintenance.\\

Après cette phase d'analyse, nous nous sommes attelés à l'architecture de \texttt{SourceCode} à l'aide d'un patchwork présentant les fonctionnalités essentielles de l'application côté \gls{frontend} et de schémas UML pour la base de données.\\

Nous avons ensuite pu travailler sur l'implémentation de \texttt{SourceCode} pour intégrer les fonctionnalités listées dans l'analyse fonctionnelle. Cette phase s'est déroulée durant quasiment toute l'année académique, sous l'oeil avisé de nos promoteurs pour garantir une cohérence dans notre travail.\\

L'étape finale fut la validation du projet, où nous avions convié des utilisateurs pour tester l'application dans son entièreté. Nous voulions nous assurer que notre mémoire fasse sens et soit utile pour la problématique que nous visons. Suite aux diverses remarques, nous avons pu effectuer une dernière itération de développement afin de prendre en considération les commentaires reçus.

\section*{Contribution}

Au travers de ce mémoire, nous espérons avoir contribué à la problématique des \glspl{resinfo} partagées en proposant une solution impliquant recherche et développement dans le domaine. Aujourd’hui et plus que jamais, l'informatique se veut de plus en plus accessible, et pour répondre à cette demande grandissante, le partage de ressources, à l'instar des \gls{oer} semble répondre à ce besoin en fournissant une banque de données qualitative et accessible publiquement.

\section*{Plan}

Nous allons d'abord définir les besoins clés d'une application de partage de \glspl{resinfo} en analysant une série de plateformes web existantes (cf. chapitre \ref{chapter:problematique}).
Après cela, nous allons nous concentrer sur une analyse plus poussée de notre plateforme en y relatant les fonctionnalités à intégrer, les besoins à satisfaire pour que la plateforme puisse être utilisée par le grand public, ainsi que les contraintes qui ont été imposées (cf. chapitre \ref{chapter:analyseDesBesoins}).\\

Nous allons par la suite expliquer notre solution à la problématique du partage de \glspl{resinfo} en détaillant l'architecture de \texttt{SourceCode} aussi bien du côté \gls{frontend} que du \gls{backend} (cf. chapitre \ref{chapter:solution}).
Pour poursuivre, nous prendrons du recul sur le projet en évoquant la phase de validation et les métriques (cf. chapitre \ref{section:validation}). 
Enfin, nous discuterons du futur de la plateforme à travers un roadmap de fonctionnalités à intégrer pour une future itération de la plateforme (cf. chapitre \ref{chapter:pourAllerPlusLoin}).

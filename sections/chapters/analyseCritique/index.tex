\chapter{Analyse critique}
\label{section:validation}

Dans ce chapitre-ci, nous nous efforçons d'avoir un regard critique sur \texttt{SourceCode}. Par la validation du projet (test avec des utilisateurs externes, test de la qualité du code) et les différentes métriques du projet, nous voulons nous assurer que notre prototype ait un sens et contribue utilement à la problématique des \glspl{resinfo}.

\import{sections/chapters/analyseCritique/}{validationExterne}

\import{sections/chapters/analyseCritique/}{validationInterne}

\section{Conclusion}

À travers ce chapitre, nous avons pu constater si ce projet tient la route à son stade actuel.\\

La séance de validation nous a été extrêmement bénéfique, car les remarques étaient constructives. 
Cela nous conforte dans l'idée que notre plateforme a des chances de devenir mature et utilisable par une communauté. 
Quoi qu'il en soit, notre projet est \textit{Open Source} et pourra donc toujours évoluer avec sa communauté. 
C'est d'ailleurs notre plus grand souhait pour cette plateforme.
La qualité technique du code est assurée par les différentes métriques énoncés précédemment (cf. section \ref{section:codeMetrics}).\\

Au niveau du planning, il semble que nous ayons respecté ce qui était initialement mis en place au vu de nos graphes d'activité. 
Nous avons tenté d'être constants tout au long du projet, car ce dernier demande beaucoup d'investissement en matière de recherche et de développement. 
Il semblerait que cela ait porté ses fruits.
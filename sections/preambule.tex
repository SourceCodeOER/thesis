\chapter*{Préambule}
\addcontentsline{toc}{chapter}{Préambule}

% Pour introduire les GROSSES objectifs du mémoire, en y incluant les sources sur github - de préférence sur un seul compte / organisation

Comme l'illustre de nombreux médias, l'informatique se révèle être une discipline prometteuse enseigné par de plus en plus d'institutions pédagogiques. Toutefois, la recherche/\\création d'exercices de programmation de qualité est une tâche chronophage.
Dans ce mémoire, nous allons présenter notre solution/prototype pour pallier à ce problème. Nous commencerons par exposer la situation actuelle et les défis à relever. Nous poursuivrons par nos idées, propositions et réalisations en justifiant nos choix technologiques et d'implémentation. Nous traiterons par la suite les différentes validations réalisées sur l'application ainsi que des retours d'utilisateurs ayant contribué à l'outil. Nous conclurons ce mémoire par une analyse critique ainsi que les possibilités futures de notre solution/prototype. \\

L'ensemble de notre solution/prototype est disponible sur Github sous notre organisation spécialement dédié à ce mémoire : 
\href{https://github.com/SourceCodeOER}{SourceCodeOER} (\href{https://github.com/SourceCodeOER}{https://github.com/SourceCodeOER}).

\begin{itemize}
    \item \href{https://github.com/SourceCodeOER/sourcecode-front}{https://github.com/SourceCodeOER/sourcecode-front} pour la partie \Gls{frontend}
    \item \href{https://github.com/SourceCodeOER/sourcecode\_api}{https://github.com/SourceCodeOER/sourcecode\_api} pour la partie \Gls{api}
    \item \href{https://github.com/SourceCodeOER/cli}{https://github.com/SourceCodeOER/cli} pour la partie \Gls{cli}
    \item \href{https://github.com/SourceCodeOER/miscellaneous}{https://github.com/SourceCodeOER/miscellaneous} pour notre base de données de test (pour la validation de notre solution par des vrais utilisateurs) et notre fichier de configuration paramétrée pour déployer l'ensemble de notre prototype avec \href{https://docs.docker.com/compose/}{Docker Compose} et ce n'importe où comme à l'UCLouvain.
\end{itemize}

Notre solution/prototype est hébergée par l'UCLouvain à l'adresse suivante : \\ \href{http://tfe-dewit-yakoub.info.ucl.ac.be/}{http://tfe-dewit-yakoub.info.ucl.ac.be/} \\

Si vous désirez découvrir notre solution/prototype dans sa totalité, voici les identifiants du compte du super administrateur : 
\begin{itemize}
    \item Adresse email : yolo24@uclouvain.be
    \item Mot de passe : API4LIFE
\end{itemize}
Nous vous demandons simplement de ne pas supprimer les données présentes sur la version en ligne afin de laisser la possibilité à autrui de découvrir cet outil dans son ensemble.


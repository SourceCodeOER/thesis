\chapter*{Préambule}
\addcontentsline{toc}{chapter}{Préambule}

% Pour introduire les GROSSES objectifs du mémoire, en y incluant les sources sur github - de préférence sur un seul compte / organisation

Présentation sommaire du mémoire et ses gros objectifs : TODO \\

L'ensemble de notre solution/prototype est disponible sur Github sous notre organisation spécialement dédié à ce mémoire : 
\href{https://github.com/SourceCodeOER}{SourceCodeOER} (\href{https://github.com/SourceCodeOER}{https://github.com/SourceCodeOER}).

\begin{itemize}
    \item \href{https://github.com/SourceCodeOER/sourcecode-front}{https://github.com/SourceCodeOER/sourcecode-front} pour la partie \Gls{frontend}
    \item \href{https://github.com/SourceCodeOER/sourcecode\_api}{https://github.com/SourceCodeOER/sourcecode\_api} pour la partie \Gls{api}
    \item \href{https://github.com/SourceCodeOER/cli}{https://github.com/SourceCodeOER/cli} pour la partie \Gls{cli}
    \item \href{https://github.com/SourceCodeOER/miscellaneous}{https://github.com/SourceCodeOER/miscellaneous} pour notre base de données de test (pour la validation de notre solution par des vrais utilisateurs) et notre fichier de configuration paramétrée pour déployer l'ensemble de notre prototype avec \href{https://docs.docker.com/compose/}{Docker Compose} et ce n'importe où comme à l'UCLouvain.
\end{itemize}

Notre solution/prototype est hébergée par l'UCLouvain à l'adresse suivante : \href{http://tfe-dewit-yakoub.info.ucl.ac.be/}{http://tfe-dewit-yakoub.info.ucl.ac.be/} \\

Si vous désirez découvrir notre solution/prototype dans sa totalité, voici les identifiants du compte du super administrateur : 
\begin{itemize}
    \item Adresse email : yolo24@uclouvain.be
    \item Mot de passe : API4LIFE
\end{itemize}
Nous vous demandons simplement simplement de ne pas supprimer les données présentes sur la version en ligne afin de laisser la possibilité à autrui de découvrir cet outil dans son ensemble.


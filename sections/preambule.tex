\chapter*{Résumé}
\addcontentsline{toc}{chapter}{Résumé}

% Pour introduire les GROSSES objectifs du mémoire, en y incluant les sources sur github - de préférence sur un seul compte / organisation
Ce travail de fin d'études a été réalisé dans le cadre de notre Master en Sciences Informatiques à l'EPL de l'UCLouvain, durant l'année académique 2019-2020.\\

De nos jours, l'informatique se révèle être une discipline prometteuse enseignée par un nombre grandissant d'institutions pédagogiques. Toutefois, l'obtention d'exercices de programmation qualitatifs et variés pour constituer son cursus demeure un premier obstacle, car cela implique l'existence d'une plateforme fiable mêlant confort de recherche et référencement cohérent des \glspl{resinfo}. 
Dans ce mémoire, nous allons présenter notre solution pour pallier ce problème par le biais de notre application web.\\

Nous commencerons par exposer la situation actuelle et les défis à relever. Nous poursuivrons par nos idées, propositions et réalisations en justifiant nos choix technologiques et d'implémentation. Nous traiterons par la suite les différentes validations réalisées sur l'application ainsi que des retours d'utilisateurs ayant contribué à l'outil. Nous conclurons ce mémoire par une analyse critique ainsi que les possibilités futures de notre prototype. \\

L'ensemble de notre prototype est disponible sur GitHub sous un compte spécialement dédié à ce mémoire : 
\href{https://github.com/SourceCodeOER}{SourceCodeOER} (\href{https://github.com/SourceCodeOER}{https://github.com/SourceCodeOER}).

\begin{itemize}
    \item \href{https://github.com/SourceCodeOER/sourcecode-front}{https://github.com/SourceCodeOER/sourcecode-front} pour la partie \gls{frontend}
    \item \href{https://github.com/SourceCodeOER/sourcecode\_api}{https://github.com/SourceCodeOER/sourcecode\_api} pour la partie \Gls{api}
    \item \href{https://github.com/SourceCodeOER/cli}{https://github.com/SourceCodeOER/cli} pour la partie \Gls{cli}
    \item \href{https://github.com/SourceCodeOER/miscellaneous}{https://github.com/SourceCodeOER/miscellaneous} pour notre base de données de test (pour la validation de notre prototype par de vrais utilisateurs) et nos fichiers de configuration pour déployer l'ensemble de notre prototype en mode production/démonstration avec \href{https://docs.docker.com/compose/}{Docker Compose} et ce n'importe où comme à l'UCLouvain.
    \item \href{https://github.com/SourceCodeOER/thesis}{https://github.com/SourceCodeOER/thesis} pour les sources du présent mémoire
\end{itemize}

Notre prototype est hébergée par l'UCLouvain en mode démonstration à l'adresse suivante : \href{http://tfe-dewit-yakoub.info.ucl.ac.be/}{http://tfe-dewit-yakoub.info.ucl.ac.be/} \\

\iffalse
Si vous désirez découvrir notre prototype dans sa totalité, voici les identifiants du compte du super administrateur : 
\begin{itemize}
    \item Adresse email : yolo24@uclouvain.be
    \item Mot de passe : API4LIFE
\end{itemize}
Nous vous demandons simplement de ne pas supprimer les données présentes sur la version en ligne afin de laisser la possibilité à autrui de découvrir cet outil dans son ensemble.
\fi